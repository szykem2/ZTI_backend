\documentclass[11pt, titlepage]{article}
\usepackage[polish]{babel}
\usepackage[T1]{fontenc}
\usepackage[utf8]{inputenc}
\usepackage{graphicx}
\usepackage{listing}
\usepackage{float}

\title{System zarządzania projektami\\Dokumentacja}
\author{unknown}
\date{\today}

\begin{document}
\maketitle

\section{Wstęp}
\hspace{11pt} W ramach projektu wykonany został system zarządzania projektami inspirowany systemami IBM RTC oraz Atlassian JIRA. Jest to uproszczona wersja takiego systemu dająca podstawową ich funkcjonalność.

Założenia projektu:
\begin{itemize}
\item Możliwość rejestracji użytkownika w systemie,
\item Obsługa autoryzacji oraz uwierzytelniania,
\item Każdy z użytkowników ma możliwość założenia nowego projektu. Taki użytkownik automatycznie zostaje administratorem projektu,
\item Użytkownik ma możliwość wyszukania projektu oraz poproszenia o dostęp do niego,
\item Użytkownik ma możliwość tworzenia elementu pracy przypisanego do projektu jeżeli sam jest autoryzowany do pracy wewnątrz projektu,
\item Użytkownik ma możliwość usunięcia elementu pracy,
\item Użytkownik ma możliwość komentowania elementu pracy,
\item Administrator projektu ma możliwość dodawania oraz usuwania użytkowników,
\item Administrator projektu ma możliwość akceptacji lub odrzucenia próśb o dołączenie do projektu,
\item Administrator projektu ma możliwość usunięcia całego projektu,
\end{itemize}

\break

\section{Baza danych}
\hspace{11pt} W projekcie została wykorzystana baza danych IBM DB2 udostępniona w chmurze IBM Cloud.

Struktura bazy danych:
\begin{figure}[H]
\caption{Diagram ERD bazy danych użytej w projekcie}
\includegraphics[width=\textwidth]{database}
\end{figure}

Opis tabel:
\begin{itemize}
\item Users - tabela przechowująca dane o użytkownikach,
\item Comments - tabela przechowująca komentarze,
\item ItemStatus - tabela przechowująca dostępne statusy elementów pracy
\item ItemType - tabela przechowująca dostępne typy elementów,
\item Requests - tabela przechowująca prośby o dołączenie do projektów,
\item Admins - tabela przechowująca administratorów projektów,
\item Item - tabela przechowująca wszystkie elementy,
\item Projects - tabela przechowująca wszystkie projekty.
\end{itemize}

Udostępniony został również skrypt w języku SQL, który tworzy bazę danych oraz dodaje rekordy stałe, czyli wypełnia tabele ItemType oraz ItemStatus.

\section{Klient}
\section{Serwer}
\section{Informacje wdrożeniowe}
\section{Działanie aplikacji}
\section{Podsumowanie}
\end{document}