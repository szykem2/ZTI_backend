\documentclass[11pt, titlepage]{article}
\usepackage[polish]{babel}
\usepackage[T1]{fontenc}
\usepackage[utf8]{inputenc}
\usepackage{graphicx}
\usepackage{listing}
\usepackage{float}
\usepackage{changepage}
\usepackage{url}
\usepackage{hyperref}
\usepackage{listings}

\title{System zarządzania projektami\\Dokumentacja}
\author{unknown}
\date{\today}

\begin{document}
\maketitle

\section{Wstęp}
\hspace{11pt} W ramach projektu wykonany został system zarządzania projektami inspirowany systemami IBM RTC oraz Atlassian JIRA. Jest to uproszczona wersja takiego systemu dająca podstawową ich funkcjonalność.

Założenia projektu:
\begin{itemize}
\item Możliwość rejestracji użytkownika w systemie,
\item Obsługa autoryzacji oraz uwierzytelniania,
\item Każdy z użytkowników ma możliwość założenia nowego projektu. Taki użytkownik automatycznie zostaje administratorem projektu,
\item Użytkownik ma możliwość wyszukania projektu oraz poproszenia o dostęp do niego,
\item Użytkownik ma możliwość tworzenia elementu pracy przypisanego do projektu jeżeli sam jest autoryzowany do pracy wewnątrz projektu,
\item Użytkownik ma możliwość usunięcia elementu pracy,
\item Użytkownik ma możliwość komentowania elementu pracy,
\item Administrator projektu ma możliwość dodawania oraz usuwania użytkowników,
\item Administrator projektu ma możliwość akceptacji lub odrzucenia próśb o dołączenie do projektu,
\item Administrator projektu ma możliwość usunięcia całego projektu,
\end{itemize}

\break

\section{Baza danych}
\hspace{11pt} W projekcie została wykorzystana baza danych IBM DB2 udostępniona w chmurze IBM Cloud.

Struktura bazy danych:
\begin{figure}[H]
\caption{Diagram ERD bazy danych użytej w projekcie}
\begin{adjustwidth}{-3cm}{}
\includegraphics[width=1.5\textwidth]{database}
\end{adjustwidth}
\end{figure}

Opis tabel:
\begin{itemize}
\item Users - tabela przechowująca dane o użytkownikach,
\item Comments - tabela przechowująca komentarze,
\item ItemStatus - tabela przechowująca dostępne statusy elementów pracy
\item ItemType - tabela przechowująca dostępne typy elementów,
\item Requests - tabela przechowująca prośby o dołączenie do projektów,
\item Admins - tabela przechowująca administratorów projektów,
\item Item - tabela przechowująca wszystkie elementy,
\item Projects - tabela przechowująca wszystkie projekty.
\end{itemize}

Udostępniony został również skrypt w języku SQL, który tworzy bazę danych oraz dodaje rekordy stałe, czyli wypełnia tabele ItemType oraz ItemStatus.

\section{Klient}
\hspace{11pt} Klient został wykonany jako strona internetowa. Komunikacja z serwerem odbywa się poprzez obiekt XHR.

Do stworzenia interfejsu wykorzystany został framework W3.CSS dostępny pod adresem:
\begin{center}
\url{https://www.w3schools.com/w3css/default.asp}
\end{center}

Klient znajduje się pod adresem:
\begin{center}
\url{http://orion.fis.agh.edu.pl/~4kubasiak/zti/projekt/}
\end{center}

Repozytorium z kodem klienta znajduje się pod adresem:
\begin{center}
\url{https://github.com/szykem2/ZTI}
\end{center}

\break

\section{Serwer}
\hspace{11pt} Serwer wykorzystuje architekturę REST w celu komunikacji. Do jego utworzenia została wykorzystana Java EE. Do komunikacji z bazą danych użyta została technologia JPA.

Wykorzystane zostało również programowanie aspektowe w języku AspectJ. W tej technologii zrealizowana została funkcjonalność logowania zmian zachodzących na serwerze.

W celu uwierzytelniania oraz autoryzacji wykorzystana została technologia JSON Web Token, zaimplementowana w w bibliotece JJWT dostępnej pod adresem:
\begin{center}
\url{https://github.com/jwtk/jjwt}
\end{center}

Po poprawnym uwierzytelnieniu, do klienta zwracany jest token, który następnie musi być przesyłany w nagłówku żądania aby przeprowadzić autoryzację.

W celu przechowywania danych uwierzytelniających wykorzystana została technologia JNDI.

Serwer został wdrożony w chmurze IBM Cloud pod adresem:
\begin{center}
\url{SK-project.eu-gb.mybluemix.net/}
\end{center}

Repozytorium z kodem serwera znajduje się pod adresem:
\begin{center}
\url{https://github.com/szykem2/ZTI_backend}
\end{center}

Udostępniony został również plik RAML przedstawiający interfejs programistyczny aplikacji:
\begin{center}
\begin{adjustwidth}{-0.0cm}{}
\url{http://orion.fis.agh.edu.pl/~4kubasiak/zti/projekt/server-api.raml}
\end{adjustwidth}
\end{center}


\break

Diagram UML projektu:
\begin{figure}[H]
\caption{Diagram klas struktury projektu}
\begin{adjustwidth}{-3cm}{}
\includegraphics[width=1.5\textwidth]{uml}
\end{adjustwidth}
\end{figure}

\section{Informacje wdrożeniowe}

\hspace{11pt} Tu będzie opis jak zdeployować.

\section{Działanie aplikacji}
\hspace{11pt} W tym punkcie opisana jest funkcjonalność aplikacji klienta. 

Po załadowaniu strony widoczny jest następujący widok. Umożliwia on zalogowanie, rejestrację przejście do dokumentacji projektu oraz dokumentacji kodu, wygenerowanej przez narzędzie JavaDoc.

\begin{figure}[H]
\begin{adjustwidth}{-3cm}{}
\includegraphics[width=1.5\textwidth]{index}
\end{adjustwidth}
\end{figure}

Kliknięcie przycisku \textbf{Sign up} powoduje wyświetlenie poniższego elementu.

\begin{figure}[H]
\includegraphics[width=\textwidth]{register}
\end{figure}

Udostępnia on funkcjonalność rejestracji do systemu. Podanie wszystkich danych jest wymagane aby się zarejestrować.

Walidacja adresu email odbywa się w oparciu o wyrażenie regularne:
\begin{adjustwidth}{-4cm}{}
\begin{lstlisting}
^[\w!#$%&'*+/=?`{|}~^-]+(?:\.[\w!#$%&'*+/=?`{|}~^-]+)*@(?:[a-zA-Z0-9-]+\.)+[a-zA-Z]{2,6}$
\end{lstlisting}
\end{adjustwidth}

Nazwa użytkownika musi się składać z od 5 do 30 znaków.

Hasło musi się składać z od 5 do 30 znaków, zawierać przynajmniej jedną wielką i małą literę oraz przynajmniej jedną cyfrę lub znak specjalny.

\break

Kliknięcie przycisku \textbf{Sign in} powoduje wyświetlenie poniższego elementu.

\begin{figure}[H]
\includegraphics[width=\textwidth]{login}
\end{figure}

Element ten udostępnia funkcjonalność logowania do systemu. Jeżeli uwierzytelnienie się powiedzie, zwracany jest token, który następnie jest wysyłany przez klienta w nagłówku żądania w celu autoryzacji i uwierzytelniania.

\break

Po zalogowaniu się do systemu oczom użytkownika ukazuje się następujące okno:

\begin{figure}[H]
\begin{adjustwidth}{-3cm}{}
\includegraphics[width=1.5\textwidth]{home}
\end{adjustwidth}
\end{figure}

Okno udostępnia funkcjonalność utworzenia nowego projektu, wylogowania się, złożenia prośby o dołączenie do projektu, oraz, po otwarciu paska nawigacyjnego wybór projektu z listy projektów, do których użytkownik ma dostęp.

\break

Użytkownik może otworzyć pasek nawigacyjny poprzez kliknięcie w przycisk w lewym, górnym rogu. Po jego otworzeniu strona wygląda następująco:

\begin{figure}[H]
\begin{adjustwidth}{-3cm}{}
\includegraphics[width=1.5\textwidth]{navbar}
\end{adjustwidth}
\end{figure}

Z dostępnych projektów można wybrać jeden, którego zawartość zostanie pobrana z serwera. Dodany został również dynamiczny filtr projektów, który powoduje wyświetlenie tylko tych projektów, które pasują do wprowadzonego wzorca.

\break

Po wybraniu projektu ukazuje się następujący widok:

\begin{figure}[H]
\begin{adjustwidth}{-3cm}{}
\includegraphics[width=1.5\textwidth]{non_admin_view}
\end{adjustwidth}
\end{figure}

Jest to widok dla użytkownika, który nie jest administratorem. Ma on dostęp do informacji o projekcie, możliwość utworzenia nowego elementu pracy, widok użytkowników autoryzowanych do wyświetlenia tego projektu oraz administratorów projektu.

Wyświetlone są również wszystkie elementy przypisane do tego projektu wraz z podstawowymi informacjami o nich.

Zaimplementowany jest również filtr elementów, który wyświetla wszystkie elementy,  do których pasuje podany wzorzec. Filtr odnosi się do dowolnego z pól, nie tylko nazwy elementu.

Klikając na wiersz tabeli użytkownik zostanie przekierowany do widoku elementu pracy, gdzie ma możliwość edycji, usuwania oraz komentowania.

\break

Widok projektu przeznaczony dla administratora:

\begin{figure}[H]
\begin{adjustwidth}{-3cm}{}
\includegraphics[width=1.5\textwidth]{admin_view}
\end{adjustwidth}
\end{figure}

Widać, że istnieją pewne zmiany. Dodana jest możliwość usuwania projektu, dodawania użytkowników, usuwania użytkowników, dodawania administratorów oraz obsługi próśb o dołączenie. Administrator ma możliwość akceptacji oraz odrzucenia prośby o dołączenie.

Wszystkie pozostałe elementy strony działają tak samo jak dla zwykłego użytkownika.

\break

W przypadku składania prośby o dostęp otwierany jest poniższy element:

\begin{figure}[H]
\includegraphics[width=\textwidth]{requesting}
\end{figure}

Użytkownik ma możliwość wyboru projektu, do którego chce mieć dostęp. Po kliknięciu rozwija się opis oraz przycisk, po kliknięciu którego zostanie wysłane żądanie do serwera aby dodać użytkownika do użytkowników proszących o dostęp.

\break

Chcąc utworzyć nowy projekt użytkownik musi wypełnić poniższy element:

\begin{figure}[H]
\includegraphics[width=\textwidth]{new_project}
\end{figure}

Należy wypełnić oba pola, w których podaje się unikalną nazwę projektu oraz krótki opis co dany projekt zawiera.

\break

Widok elementu pracy:

\begin{figure}[H]
\begin{adjustwidth}{-3cm}{}
\includegraphics[width=1.5\textwidth]{item}
\end{adjustwidth}
\end{figure}

Zaprezentowany został tylko podgląd elementu. Pozostałe dwa tryby, czyli edycja oraz tworzenie nowego wyglądają identycznie, poza faktem edytowalności pól. Możliwość edycji uzyskuje się po kliknięciu przycisku \textbf{EDIT}.

Poza podglądem oraz edycją użytkownik ma możliwość dodawania komentarzy. Możliwość ta jest całkowicie zablokowana podczas tworzenia nowego elementu.

\end{document}